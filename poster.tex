\documentclass[a0paper,portrait]{baposter}

% Packages nécessaires
\usepackage[utf8]{inputenc}
\usepackage[T1]{fontenc}
\usepackage[french]{babel}
\usepackage{graphicx}
\usepackage{xcolor}
\usepackage{tikz}
\usepackage{multicol}
\usepackage{adjustbox}
\usepackage{qrcode}
% \usepackage{fontawesome5} % Removed due to compatibility issues
\usepackage{lmodern}
\usepackage[a-1b]{pdfx}

% Configuration des couleurs
\definecolor{c4dtblue}{RGB}{0,102,179}
\definecolor{gnugengray}{RGB}{85,85,85}
\definecolor{lightblue}{RGB}{73,180,230}
\definecolor{darkgray}{RGB}{64,64,64}
\definecolor{lightgray}{RGB}{240,240,240}
\definecolor{darkgreen}{RGB}{160, 200, 160}
\definecolor{darkblue}{RGB}{90, 180, 200}

%%  - \tiny < \scriptsize < \footnotesize < \small < \normalsize < \large < \Large < \LARGE < \huge < \Huge < \HUGE

% Configuration du poster
\begin{document}
\begin{poster}{
  % Options du poster
  columns=2,
  colspacing=1em,
  bgColorOne=white,
  bgColorTwo=white,
  borderColor=c4dtblue,
  headerColorOne=c4dtblue,
  headerColorTwo=lightblue,
  headerFontColor=white,
  boxColorOne=white,
  boxColorTwo=lightgray,
  textborder=rounded,
  eyecatcher=true,
  headerheight=0.08\textheight,
  headershape=rounded,
  headerfont=\huge\bf\textsf,
  textfont={\setlength{\parindent}{0em}\large},
  boxshade=plain,
  background=plain,
  linewidth=2pt
}{
\begin{minipage}{5cm}
\includegraphics[height=6em]{assets/logos/C4DT_logo.png}%
\end{minipage}
}
{Confiance Numérique}
{Comment préserver votre vie privée sur internet}
{
\begin{minipage}{5cm}
\includegraphics[width=5cm]{assets/logos/gnugen_logo.png} \\
\includegraphics[width=5cm]{assets/logos/digiges_logo.png}
\end{minipage}
}

% Contenu principal du poster
\headerbox{Êtes-vous client ou produit ?}{name=intro,column=0,row=0}{
  \vspace{0.5em}
  \begin{center}
  \includegraphics[width=0.6\textwidth]{assets/youre-the-product.jpg}
  \end{center}

  \textbf{\Large Votre vie privée : un droit fondamental}

  \textit{\Large Constitution Fédérale Suisse, Article 13 :}

  \textcolor{darkgray}{\large "Toute personne a droit au respect de sa vie privée et familiale, de son domicile, de sa correspondance et des relations qu'elle établit par la poste et les télécommunications."}

  \vspace{1em}

  \textbf{\color{c4dtblue}\LARGE Reprenez le contrôle !}
  \begin{itemize}
  \item{ Des alternatives libres existent}
  \item{ Protégez-vous sans sacrifier la simplicité}
  \item{ Vos données valent de l'or}
  \end{itemize}
}

\headerbox{\huge Les enjeux}{name=stats,column=0,below=intro}{
  \textbf{\color{red}\Large E-mail :}
  \begin{itemize}
  \item{ 251 millions d'emails/minute}
  \item{ \textbf{3,4 milliards} d'emails de phishing quotidiens}
  \end{itemize}

  \vspace{1em}

  \textbf{\color{red}\Large Surveillance :}
  \begin{itemize}
  \item{ Profilage comportemental}
  \item{ Données vendues aux gouvernements}
  \item{ Ciblage publicitaire invasif}
  \end{itemize}

  \vspace{1em}

  \textbf{\color{red}\Large Monopolisation :}
  \begin{itemize}
  \item{ Concentration du pouvoir}
  \item{ Contrôle de l'information}
  \item{ Restriction des libertés numériques}
  \end{itemize}
}

\headerbox{Venez discuter avec nous !}{name=stats,column=0,below=stats,height=bottom}{
  \textbf{\color{c4dtblue}Qui sommes-nous ?}
  \begin{itemize}
  \item \textbf{C4DT} - Centre pour la Confiance Numérique
  \item \textbf{gnugen} - Promotion du logiciel libre à l'EPFL
  \item \textbf{DigiGes} - Partenaire pour la transition numérique
  \item \textbf{HEIG/VD} - Haute école d'ingénierie et de gestion
  \end{itemize}

  \vspace{1em}

  \textbf{\color{c4dtblue}Quelques questions pour commencer :}
  \begin{itemize}
  \item Qui s'intéresse à mes données? Je n'ai rien à cacher.
  \item Comment choisir entre sécurité et confort ?
  \item Quel niveau de sécurité me convient ?
  \item Pouvez-vous m'aider à installer une application sécurisée ?
  \end{itemize}

  \vspace{1em}

  \begin{minipage}[t]{0.7\linewidth}
    \textbf{Contacts :} \\
    \vspace{2em}
    \small
    \begin{tabular}{ll}
    \\
    C4DT & Carine Dengler (@cdengler@social.epfl.ch) \\
         & Linus Gasser (@ligasser@social.epfl.ch) \\
    GnuGen & Nils Antonovich (@antonovi:gnugen.ch) \\
         & Yaëlle Dutoit (@ydutoit:gnugen.ch) \\
         & Jonas Sulzer (@jonas:violoncello.ch) \\
    DigiGes & https://digitale-gesellschaft.ch \\
    HEIG/VD & Alexis Pinto \\
    \end{tabular}%
  \end{minipage}%
  \begin{minipage}[t]{0.28\linewidth}
    \begin{flushright}
    \small\textbf{Guide de survie numérique de GnuGen} \\
    \vspace{1em}
    \qrcode[height=2.5cm]{https://gnugen.ch/fr/guides/}
    \footnotesize{gnugen.ch/fr/guides/}
    \end{flushright}
  \end{minipage}
}

%
% Second Column
%

\newcommand{\qrspace}{\vspace{1em}}
\newcommand{\qrcodeurl}[3][0.75cm]{
    \centering
    \begin{tikzpicture}
        \node(qrcode){\qrcode[height=2.2cm,level=H]{https://#2}};
        \node(overlay){\includegraphics[width=#1]{assets/apps/#3.png}};
    \end{tikzpicture} \\
    \small{#2}
}

\headerbox{Communications en Ligne}{name=communication,column=1,row=0,headerColorOne=darkgreen,headerColorTwo=darkblue}{
  \begin{minipage}{0.7\linewidth}
    \textbf{\color{c4dtblue}X $\rightarrow$ Mastodon}
    \begin{itemize}
      \item Pas d'algorithme manipulateur
      \item Données non vendues aux annonceurs
      \item Contrôle total de votre fil d'actualité
    \end{itemize}
    \vspace{0.5em}
  \end{minipage}
  \begin{minipage}{0.29\linewidth}
    \qrcodeurl{mastodon.social}{mastodon}
  \end{minipage}

  \qrspace

  \noindent
  \colorbox{lightgray}{
    \begin{minipage}{0.25\linewidth}
      \qrcodeurl{signal.org}{signal}
    \end{minipage}%
    \begin{minipage}{0.7\linewidth}
      \hspace{0.5em}
      \textbf{\color{c4dtblue}WhatsApp $\rightarrow$ Signal}
      \begin{itemize}
        \item Métadonnées minimales collectées
        \item Chiffrement vérifié par des experts
        \item Pas de liens avec Meta/Facebook
      \end{itemize}
    \end{minipage}
  }

  \qrspace

  \noindent
  \begin{minipage}{0.7\linewidth}
    \textbf{\color{c4dtblue}Slack $\rightarrow$ Matrix}
    \begin{itemize}
      \item Protocole décentralisé et ouvert
      \item Chiffrement bout-à-bout par défaut
      \item Pas d'analyse de vos conversations
    \end{itemize}
  \end{minipage}%
  \begin{minipage}{0.29\linewidth}
    \qrcodeurl{app.element.io}{matrix}
  \end{minipage}
}

\headerbox{Bureautique}{name=office,column=1,below=communication,headerColorOne=darkgreen,headerColorTwo=darkblue}{
  \begin{minipage}{0.29\linewidth}
    \qrcodeurl{infomaniak.com/fr/ksuite}{kdrive}
  \end{minipage}
  \begin{minipage}{0.7\linewidth}
    \textbf{\color{c4dtblue}MS365 $\rightarrow$ K-Drive}
    \begin{itemize}
      \item Données stockées en Suisse (nLPD)
      \item N'utilise pas vos fichiers pour la pub
      \item Peut chiffrer de bout-en-bout
    \end{itemize}
    \vspace{1em}
  \end{minipage}%

  \qrspace

  \noindent
  \colorbox{lightgray}{
    \begin{minipage}{0.7\linewidth}
      \textbf{\color{c4dtblue}Outlook $\rightarrow$ Thunderbird}
      \begin{itemize}
        \item Aucune collecte de métadonnées
        \item Open source = transparence totale
        \item Fonctionne avec tous les emails
      \end{itemize}
    \end{minipage}
    \begin{minipage}{0.25\linewidth}
      \qrcodeurl{thunderbird.net}{thunderbird}
    \end{minipage}%
  }

  \qrspace

  \noindent
  \begin{minipage}{0.29\linewidth}
    \qrcodeurl{firefox.com}{firefox}
  \end{minipage}
  \begin{minipage}{0.7\linewidth}
    \textbf{\color{c4dtblue}Chrome $\rightarrow$ Firefox}
    \begin{itemize}
      \item Bloque le tracking par défaut
      \item Moteur indépendant (pas Google)
      \item Extensions anti-pub plus efficaces
    \end{itemize}
  \end{minipage}%
}

\headerbox{Sécurité}{name=security,column=1,below=office,height=bottom,headerColorOne=darkgreen,headerColorTwo=darkblue}{
  \begin{minipage}{0.7\linewidth}
    \textbf{\color{c4dtblue}Gestion de mots de passe}
    \begin{itemize}
      \item Mots de passe uniques et forts
      \item Chiffrement local de vos données
      \item Sécurisé avec un seul mot de passe
    \end{itemize}
    \vspace{0.5em}
  \end{minipage}%
  \begin{minipage}{0.29\linewidth}
    \qrcodeurl{bitwarden.com}{bitwarden}
  \end{minipage}

  \qrspace

  \noindent
  \colorbox{lightgray}{
    \begin{minipage}{0.25\linewidth}
      \qrcodeurl{ublockorigin.com}{ublockorigin}
    \end{minipage}%
    \begin{minipage}{0.7\linewidth}
      \textbf{\color{c4dtblue}Non aux pubs avec uBlock}
      \begin{itemize}
        \item Bloque trackers et publicités invasives
        \item Protection contre sites malveillants
        \item Navigation plus rapide et fluide
      \end{itemize}
    \end{minipage}
  }

  \qrspace

  \noindent
  \begin{minipage}{0.7\linewidth}
    \textbf{\color{c4dtblue}Consent-o-matic}
    \begin{itemize}
      \item Refuse automatiquement les cookies
      \item Évite le tracking publicitaire
      \item Enlève les popups
    \end{itemize}
  \end{minipage}%
  \begin{minipage}{0.29\linewidth}
    \qrcodeurl{consentomatic.au.dk}{consentomatic}
  \end{minipage}
}

\end{poster}
\end{document}
